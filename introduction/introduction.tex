\cleardoublepage
\phantomsection
 \addcontentsline{toc}{chapter}{Introduction}
\chapter*{Introduction}
Les ordinateurs et terminaux mobiles ont connu durant les dernières années une évolution exponentielle en termes de performance et de fonctionnalités. Cela a préparé le terrain pour des usages assez avancés de la technologie, spécifiquement dans le développement d’applications web et mobile, capables de répondre aux attentes de plus en plus grandissantes du public.

Les béninois utilisent de plus en plus les systèmes numériques, confirmant ainsi la véritable révolution technologique que vivent depuis quelques années, tous les secteurs d’activité de notre pays. Les secteurs de la culture et du tourisme en particulier sont en plein essor et présentent un avantage particulier pour l’innovation. 

Malgré ses avancées, il semble difficile d’apporter un regard neuf sur la richesse touristique béninoise. De plus, le logement au Bénin reste une difficulté, même pour les nationaux en déplacement à l'intérieur du territoire. Face à cette situation, il est important de trouver une solution adéquate permettant de répondre au mieux aux besoins du secteur du logement et du tourisme au Bénin.

C’est dans ce contexte que s’inscrit le présent projet intitulé : “Conception et développement d’une plateforme  de location court durée d’appartements entre particuliers”. Il comprend quatre (04) chapitres. Le premier chapitre est consacré à la description du cadre de notre stage, le deuxième au contexte de l’étude et à l’état de l’art. Le troisième chapitre intitulé « Analyse, Conception et Choix Techniques » présente une proposition de solution, le choix de la méthode d'analyse, l’analyse fonctionnelle et conceptuelle ainsi que le choix des outils d'implémentation et l’architecture du système. Le quatrième et dernier chapitre intitulé « Réalisation de l’application », décrit les interfaces développées côté client, le serveur et la base de données de l’application.
