%%%%%%%%%%%%%%%%%%%%%%%%%%%%%%%%%Les packages%%%%%%%%%%%%%%%%%%%%%%%%%%%%%%%%%%%%%%
\usepackage[utf8]{inputenc}%           gestion des accents (source)
\usepackage[T1]{fontenc}%              gestion des accents (PDF)
\usepackage[english,french]{babel}
%\usepackage[francais]{babel}%          gestion du français
\usepackage{textcomp}%                 caractères additionnels
\usepackage{listings}%                 pour insérer du code source
\usepackage{amssymb,amsthm,amsmath,amsfonts,mathrsfs}% packages de l'AMS + mathtools
\usepackage{lmodern}%                  police de caractère
\usepackage{geometry}%                 gestion des marges
	\geometry{papersize={21cm,29.7cm}}
	\geometry{margin=2cm,bottom=2cm}
\usepackage{graphicx}%                 gestion des images
	\graphicspath{{images/}}%pour spécifier le chemin d'accès aux images
\usepackage{xcolor}%                   gestion des couleurs
\usepackage{array}%                    gestion améliorée des tableaux
\usepackage{tgpagella}%
\usepackage{calc}%                     syntaxe naturelle pour les calculs
\usepackage{titlesec}%                 pour les sections
\usepackage{titletoc}%                 pour la table des matières
\usepackage{fancyhdr}%                 pour les en-têtes
\usepackage{titling}%                  pour le titre
\usepackage{enumitem}%                 pour les listes numérotées
\usepackage[breaklinks]{hyperref}%                 gestion des hyperliens, rend actif les liens, références croisées,
\hypersetup{pdfstartview=XYZ}%         zoom par défaut
%\usepackage{abstract}%
\usepackage{stmaryrd}% 				   pour délimiteurs ouvrants et fermants
\usepackage{footmisc}%
\usepackage{multirow}%				   Fusion de cellules
\usepackage{pdfpages}%					Pour inclure des pages entières d’un PDF
\usepackage{caption}%					placement manuel d'image
%\usepackage{mathdots}%permet d’avoir des points de suspension corrects en indice ou exposant.
\usepackage{shorttoc}%                  pour la réalisation d'un sommaire
\usepackage{remreset}%
\usepackage{pgf,tikz}%             gestion des dessins  
	\usetikzlibrary{arrows}
\usepackage{eurosym}%
\usepackage{soul}%
\usepackage{bookman}
\usepackage{ wrapfig}
\usepackage{ulem}
\usepackage{ setspace}
\usepackage{ layout}
\usepackage{lipsum}% juste utile ici pour générer du faux texte}
%\usepackage{mwe}%juste utile ici pour générer de fausses images
\usepackage[Rejne]{fncychap}%pour de jolis titres de chapitres voir la doc pour d'autres styles.
\usepackage{soul}
\usepackage{float}
%%%%%%%%%%%%%%%%%%%style main%%%%%%%%%%%%%%%%%%%%%%%%%%%%%%%%%%%%
        \fancypagestyle{mainmatter} {}
                \fancyhf{}
                \fancyhead[R]{\leftmark} % Right Even
                \fancyfoot[C]{\thepage}
                %\fancyfoot[C]{\thepage \\\scriptsize Conception et développement d’une plateforme communautaire de découverte d’expérience et de location à court terme d’appartement ou de parties d’appartement entre particuliers}
               
             
\renewcommand{\headrulewidth}{1.5pt}%trait horizontal pour l'en-tête
        \renewcommand{\footrulewidth}{2pt}%trait horizontal pour les pieds de pages              
%%%%%%%%%%%%%%%%%%%%%%%%%%%%%%%%%%%%%%%%%%%%%%%%%%%%%%%%%%%%%%%%%%%%%%%%%%%%%%%%%%%%%%%%%%%%%%%%%%%%%%%%%%%%%%%%%%%%%%%%%%%%%%%%%%%%%%%%%%%%
  \setlength{\headheight}{14.2pt}% hauteur de l'en-tête
\usepackage{ makeidx} 
\usepackage[french]{nomencl}
\lstnewenvironment{ccode}[1][]
  {\lstset{language=C,numbers=left,numberstyle=\tiny,#1}}
  {}
\lstset{aboveskip=\topsep,belowskip=\topsep,
        frame=single,rulecolor=\color{black}}%,backgroundcolor=\color{blue!5}}
        
%%%%%%%%%%%%%%%%%%%style back%%%%%%%%%%%%%%%%%%%%%%%%%%%%%%%%%%%%%%%%%  
        \fancypagestyle{backmatter}{%
                \fancyhf{}%on vide les en-têtes
                \fancyfoot[C]{ \thepage}%
                \renewcommand{\headrulewidth}{1.5pt}%trait horizontal pour l'en-tête
                \renewcommand{\footrulewidth}{2pt}%trait horizontal pour les pieds de pages
                }
                
 %%%%%%%%%%%%%%%%%%%%%%%%%%%%%%%%%%%%%%%%%%%%%%%%%%%%%%%%%%%%%%%%%%
\makenomenclature
\renewcommand{\nomname}{Liste des abréviations, des sigles et des symboles}
\makeatletter
\newenvironment{abstract}{%
    \cleardoublepage
    \null\vfil
    \@beginparpenalty\@lowpenalty
    \begin{center}%
      \bfseries \abstractname
      \@endparpenalty\@M
    \end{center}}%
   {\par\vfil\null}
\makeatother
 %%%%%%%%%%%%%%%%%%%style front%%%%%%%%%%%%%%%%%%%%%%%%%%%%%%%%%%%%%%%%% 
        \fancypagestyle{frontmatter}{
                \fancyhf{}%on vide les en-têtes
                \fancyfoot[C]{ \thepage}%
                \renewcommand{\headrulewidth}{1.5pt}%trait horizontal pour l'en-tête
                \renewcommand{\footrulewidth}{2pt}%trait horizontal pour les pieds de pages
                }
%%%%%%%%%%%%%%%%%%%%%%%%%%%%index%%%%%%%%%%%%%%%%%%%%%%%%%%%%%%%%%%%%%%%
\makeindex
                \hypersetup{colorlinks,%
                citecolor=black,%
                filecolor=black,%
                linkcolor=black,%
                urlcolor=black} 
%%%%%%%%%%%%%%%%%%%%%%%%%%%%%% les environnements %%%%%%%%%%%%%%%%%%%%%%%%%%%%%%%%%%%%%%%
\newcounter{exos}[section]
 \newenvironment{exos}{\stepcounter{exos}\vspace{0.5cm}{\center{\bfseries\ul{Exercice  \theexos}}}}{\par\vspace{0.5cm}}
\newcounter{sol}[section]
 \newenvironment{sol}{\stepcounter{sol}\vspace{0.5cm}{\center{\bfseries \ul{Solution  \thesol}}}}{\par\vspace{0.5cm}} 


%%%%%%%%%%%%%%%%%%%%%%%%%%%%%%%%%%%%%%%%  setting  page and marges %%%%%%%%%%%%%%%%%%%%%%%%%

 %\linespread{1.2}%{1.4}% this is for the intervall between lines
%%%%%%%%%%%%%%%%%%%%%%%%%%%%biblio%%%%%%%%%%%%%%%%%%%%%%%%%%%%%%%%%%%%
%\usepackage[backend=biber]{biblatex}
%\addbibresource{bibliographie/biblio.bib}% pour indiquer où se trouve notre .bib
\usepackage{csquotes}% pour la gestion des guillemets français.
\usepackage{nomencl}
\makenomenclature
\renewcommand{\nomname}{Liste des abréviations, des sigles et des symboles}

%%%%%%%%%%%%%%%%%%%%%%%%%%%%%%%%%%%%%%%%%%%%%%%%%%%%%%%%%%%%%%%%%%%
\usepackage{draftwatermark}
	\SetWatermarkLightness{0.8}
	\SetWatermarkAngle{25}
	\SetWatermarkScale{2}
	\SetWatermarkFontSize{1cm}
	\SetWatermarkText{} 
	
%%%%%%%%%%%%%%%%%%%%%%%%%%%%%%%%%%%%%%%%%%%%%%%% chapter style defined %%%%%%%%%%%%%%%%%%%%%%%%%%%%%%%%%%%
\titleclass{\subsubsubsection}{straight}[\subsection]

\newcounter{subsubsubsection}[subsubsection]
\renewcommand\thesubsubsubsection{\thesubsubsection.\arabic{subsubsubsection}}

\titleformat{\subsubsubsection}
  {\normalfont\normalsize\bfseries}{\thesubsubsubsection}{1em}{}
\titlespacing*{\subsubsubsection}
{0pt}{3.25ex plus 1ex minus .2ex}{1.5ex plus .2ex}

\makeatletter

\def\toclevel@subsubsubsection{4}
\def\l@subsubsubsection{\@dottedtocline{4}{7em}{4em}}
\makeatother

\setcounter{secnumdepth}{4}
\setcounter{tocdepth}{4}
                              
%%%%%%%%%%%%%%%%%%%%%%%%%%%%%%%%%%%%%%%%%%%%%%%%%%%%%%%%%%%%%%%%%%%%%%%%%%