\chapter{Description institutionnelle d’EtriLabs}
Notre stage s’est déroulé à l’ONG EtriLabs Cotonou. Cette partie aborde l’historique de notre structure d’accueil, le rôle et les différentes activités que nous y avons effectuées ainsi que les compétences que nous avons acquises dès lors.

\section{Présentation de l'entreprise}

\subsection{Historique de l’entreprise}
Educational Technology and Research International (ETRI) opérant sous le nom commercial EtriLabs, est une Organisation Non-Gouvernementale qui se sert des Technologies de l’Information et de la Communication (TIC) pour intervenir dans plusieurs domaines. Elle a été créée en 2010 pour pallier entre autres, le retard de l’Afrique et du Bénin en particulier dans le domaine du numérique. Son objectif étant de promouvoir la culture de l’innovation et de la compétitivité en incluant toutes les couches sociales, elle a pour mission de faire la promotion des TIC au service du développement.

\subsection{Organisation de l’entreprise}
EtriLabs est présente dans deux pays d’Afrique : le Bénin et le Sénégal. Au Bénin elle est présente à Parakou et à Cotonou où se trouve son siège officiel. De façon générale, quatre niveaux hiérarchiques s’y succèdent. Nous avons donc:

\paragraph{Le Président Directeur Général}
$ $\\Le Président Directeur Général de EtriLabs est le représentant légal de l’entreprise. Il préside le conseil d’administration. En cette qualité:
\begin{itemize}
\item[\textbullet] Il définit ses objectifs généraux et décide en dernier ressort des moyens financiers, matériels et humains à mettre en œuvre dans le cadre des orientations et des décisions du conseil d’administration ;
\item[\textbullet] Il anime le comité de direction de l’entreprise et est responsable de ses résultats ;
\item[\textbullet] Il est chargé de la gestion des contrats avec les partenaires ;
\item[\textbullet] Il est enfin chargé de la culture de l’organisation.
\end{itemize}

\paragraph{Le Directeur Technique}
$ $\\Le directeur technique supervise l'ensemble des projets de l’entreprise, depuis l'étape de la conception jusqu'au support. Il est concrètement chargé de :
\begin{itemize}
\item[\textbullet] mesurer les retours sur investissement ;
\item[\textbullet] définir ou de réorienter une nouvelle stratégie d'entreprise en fonction des résultats ;
\item[\textbullet] établir des programmes de recherche et de développement puis d’apporter des idées quant aux nouveaux produits et procédés ;
\item[\textbullet] coordonner le développement des produits en concordance avec la réglementation en matière de qualité de sécurité et d'environnement.
\end{itemize}

\paragraph{Le Directeur des Programmes}
$ $\\Le Directeur des Programmes de EtriLabs, supervise la coordination et l'administration de tous les aspects des programmes de EtriLabs, y compris la planification, l'organisation, la dotation, la conduite et le contrôle des activités des programmes.
Il gère l'implantation financière, logistique et journalière des programmes WHISPA, TEKLIONS et Learn2Code.

\paragraph{L’Agent des Programmes} 
$ $\\L’Agent des Programmes de EtriLabs Cotonou est chargée de :
\begin{itemize}
\item[\textbullet] Construire et développer des relations avec les partenaires stratégiques et la communauté ;
\item[\textbullet] Gérer, planifier, coordonner et superviser les activités, les événements, les séances de formation et les programmes;
\item[\textbullet] Contribuer à la conception et au développement de propositions de projets.
\end{itemize}

EtriLabs Cotonou offre plusieurs services dont l’incubation et l’accélération de startups. Les équipiers de ces startups représentent en majorité son personnel. Au nombre de ces startups nous avons : Pikiz, Intside, Queezly, Chaperone, ClassAction, HappierCo, Sewema, BeninMaison, Mentorat Club, Socializer et Botamp.
\\Ces dernières ont chacune une structure propre. La taille des équipes varie de deux à six équipiers desquels on dénombre des développeurs et d’autres chargés du marketing.

\subsection{Activités de l’entreprise}
EtriLabs offre plusieurs services que sont:

\begin{itemize}
\item[\textbullet] Accélération et l'incubation de startups ;
\item[\textbullet] Solutions technologiques innovantes aux entreprises ;
\item[\textbullet] Événements, formations et coworking.
\end{itemize}


\paragraph{Accélération et incubation de startups}
$ $\\Les programmes d'accélération et d’incubation de EtriLabs apportent un soutien technique, un mentorat de haute qualité, des bureaux, et un financement pour accompagner les entrepreneurs.
Depuis 2014, EtriLabs a accéléré plus de dix (10) startups qui fournissent des produits de classe mondiale à des milliers de clients à travers le monde.

\paragraph{Solutions aux Entreprises}
$ $\\EtriLabs propose aux entreprises, particuliers, organisations non gouvernementales et gouvernements des solutions sur mesure pour atteindre leurs objectifs. Leurs prestations externalisées offrent une grande souplesse pour l’organisation d’événements, de même que la conception et la mise en œuvre de stratégies de communication – marketing.
\newpage
\paragraph{Évènements}
$ $\\EtriLabs organise des séminaires, hackathons et soirées thématiques pour connecter, former et informer la communauté. Ces événements constituent un creuset de partage de connaissances et d’expériences et facilitent le networking entre les investisseurs, les développeurs, les éducateurs, les décideurs, entrepreneurs et autres parties prenantes pour créer les conditions propices à l’innovation.

\paragraph{Formations}
$ $\\A travers ces différents programmes de formations, EtriLabs entend impacter la communauté et créer une génération de « success stories » entrepreneuriales dans le numérique. Elle détient une vaste expérience dans la formation et a lancé avec succès deux programmes: le camp Learn2Code pour susciter chez les enfants et adolescents la passion pour la programmation informatique et le programme WHISPA\footnote{Women High Impact Startup Preparation Academy} qui forme les femmes à l’entrepreneuriat numérique.

\paragraph{Coworking} 
$ $\\L’offre de coworking d'EtriLabs permet aux entrepreneurs, développeurs ou même artistes, de travailler dans un environnement moderne, agréable et numérique où ils ont en partage toutes les ressources d’un bureau typique sans les coûts exorbitants qui viennent avec.

\section{Objectifs du stage}
Notre immersion dans le monde professionnel nous a permis de confronter nos connaissances théoriques à la pratique. Ainsi notre objectif était de :
\begin{itemize}
\item[\textbullet] Comprendre le monde professionnel ;
\item[\textbullet] Apprendre de nouveau outils de programmation ;
\item[\textbullet] Apprendre de nouvelles technologies ;
\item[\textbullet] Comprendre ce qu’est une startup et comment elle fonctionne.
\end{itemize}

\subsection{Activités effectuées}
Stagiaire à EtriLabs, nous avons eu à nous familiariser avec un bon nombre d’opérations sous la tutelle de mentors raffinés. Cette partie met l’accent sur les tâches effectuées au sein de l’entreprise.
\\Dès le debut, notre mentor nous a guidé dans l’apprentissage et l’utilisation d’outils simples mais indispensables pour tout développeur. Il s’agit de :
\begin{itemize}
\item[\textbullet] Git ;
\item[\textbullet] Github et Gitlab ;
\item[\textbullet] Trello ;
\item[\textbullet] Bootstrap.
\end{itemize}

Par la suite, nous avons fait une immersion complète dans l'écosystème JavaScript : JS pur, Node.Js, NPM, TypeScript, React.
\\Nous avons également appris l’utilisation d’une base de données dite NoSQL : MongoDB.
\\Cet apprentissage nous a amené à développer un outil de sauvegarde et de restauration en ligne dédié aux bases de données MongoDB. Le développement de cette plateforme disponible a l’adresse \url{www.mongomanager.space} pris le gros de nos six mois de stage.
\\Enfin, nous avons expérimenté l’utilisation de quelques frameworks de développement d’applications hybrides : NativeScript et ReactNative.
\newpage
\subsection{Acquis de stage}
Nos six mois à EtriLabs nous ont été très bénéfiques. Nous avons développés grâce à l’observation et aux activités effectuées des compétences en conception et déploiement d’applications JavaScript, ainsi qu’en développement d’application mobiles hybrides.
\\Avec plus de conviction nous croyons que l’utilisation de Git et de Github est indispensable pour les développeurs. De plus, avec le JavaScript, nous disposons maintenant d'un écosystème incroyablement riche et dynamique, et d'un paradigme "universel" pour construire les applications de demain, web ou natives.

\subsection{Difficultés rencontrées}
Dans un environnement nouveau, l’homme apprend à s’adapter. Au cours de cette adaptation, il rencontre nécessairement des obstacles. Tel a été notre cas ; au nombre donc de ces obstacles nous retenons :
\begin{itemize}
\item[\textbullet] L’indisponibilité fréquente ou la lenteur de la connexion internet ;
\item[\textbullet] L’indisponibilité fréquente de l'énergie électrique ;
\end{itemize}

Par rapport au mémoire, nous avons eu de grosses difficultés à respecter le planning initialement prévu notamment à cause du cahier des charges. En effet nous avons mi trop longtemps à définir exactement ce dernier.
\\Ainsi, nous avons dû établir des modifications au cours de la réalisation du projet, ce qui a provoqué un retard sur l’ensemble de nos prévisions. Nous aurions dû, dès le départ, et très rapidement, écarter les idées irréalisables ou les propositions irréalistes.